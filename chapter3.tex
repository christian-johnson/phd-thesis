\chapter[Black Holes]{Black Holes}\label{chapter:pbhs}
This chapter provides background information on black holes

\section{History}

\section{Theory}
Einstein's equation (in natural units, i.e. $c, h = 1$) is:
\begin{equation}
G_{\mu\nu} = 8\pi T_{\mu\nu}
\end{equation}
The unique static, spherically-symmetric solution to Einstein's equation is the Schwarszchild metric
Schwarszchild metric
\subsection{Hawking Radiation}
In the classical treatment, black holes have no temperature and do not emit blackbody radiation; intuitively, energy cannot escape a black hole because the worldlines of particles inside the event horizon are trapped within Schwarschild radius $r=2GM$. 
In the 1970s, however, it was noted that black holes appear to exhibit many of the same properties as classical thermodynamic systems.
The black hole area theorem is one example: during black hole mergers, the total black hole surface area is nondecreasing in much the same way that the Second Law of thermodynamics states that the entropy of a closed system is nondecreasing.
Further work by Bekenstein in 1972 led to the following analogies between black holes and thermodynamic quantities:
\begin{center}
\begin{tabular}{ccc}
Entropy & $\longleftrightarrow$ & Surface Area\\
Temperature & $\longleftrightarrow$ & Surface Gravity $\kappa$\\
Energy & $\longleftrightarrow$  & Mass\\
\end{tabular}
\end{center}
Black hole thermodynamics remains an area of active research, and recent work has added new terms to this list.
As an example, in anti-deSitter space the geometrical volume of a Schwarszchild black hole is equal to the thermodynamic volume associated with it.

However, the work of Hawking and Beckenstein in the 1970s led to the understanding that black holes are really cool.
The Hawking temperature of a black hole is given by:
\begin{equation}
T = \frac{\hbar c^3}{8\pi k_B GM}
\end{equation}

As the black hole emits radiation, it loses mass and becomes hotter.
This process continues until the entirety (or nearly so- see Section X) of the black hole mass is emitted as Hawking radiation.

\section{Primordial Black Holes}
Black holes formed via stellar collapse cannot be smaller than approximately 2 solar masses.
\subsection{Formation Mechanisms}
See Halzen et al etc
\subsection{Constraints on Dark Matter}
Stuff from Carr et al, Kusenko et al

\subsection{Prospects for detection}
For a description of the $\gamma$-ray spectrum of small black holes, see Appendix \ref{chapter:pbh_spectrum}.