\chapter[Black Holes]{Black Holes}
This chapter provides background information on black holes

\section{History}

\section{Theory}
Einstein's equation (in natural units, i.e. $c, h = 1$) is:
\begin{equation}
G_{\mu\nu} = 8\pi T_{\mu\nu}
\end{equation}
The unique static, spherically-symmetric solution to Einstein's equation is the Schwarszchild metric
Schwarszchild metric
\subsection{Hawking Radiation}
In the classical treatment, black holes have no temperature and do not emit blackbody radiation; intuitively, energy cannot escape a black hole because the worldlines of particles inside the event horizon cannot exceed $r=2GM$. 
Interestingly, however, black holes have been shown to exhibit many of the same properties as classical thermodynamic systems.
During black hole mergers, for instance, the total black hole surface area is nondecreasing.
One can draw analogy between properties of black holes and thermodynamic quantities like so:
\begin{tabular}{ccc}


Temperature & $\longleftrightarrow$ & Surface Gravity $\kappa$\\
Entropy & $\longleftrightarrow$ & Surface Area\\
Energy & $\longleftrightarrow$  & Mass\\

\end{tabular}

However, the work of Hawking and Beckenstein in the 1970s led to the understanding that black holes are really cool.
The Hawking temperature of a black hole is given by:
\begin{equation}
T = \frac{\hbar c^3}{8\pi k_B GM}
\end{equation}

As the black hole emits radiation, it loses mass and becomes hotter.
This process continues until the entirety (or nearly so- see Section X) of the black hole mass is emitted as Hawking radiation.

\section{Primordial Black Holes}
Black holes formed via stellar collapse cannot be smaller than approximately 2 solar masses.
\subsection{Formation Mechanisms}
\subsection{Constraints on Dark Matter}
