\section{A Search for PBH Candidates in \Fermi-LAT 3FGL Catalog}
\label{sec:search}
The \Fermi-LAT surveys the entire sky approximately every three hours, and has relatively uniform exposure over long time scales (at 1 GeV over 4 years the exposure varies by approximately 40\% over the entire sky). Combined with a large effective area (approximately 1 m$^2$ between 1 GeV and 1 TeV), this makes it an ideal instrument for detecting a large number of gamma-ray point sources. 
The most complete catalog of point sources is currently the third \Fermi point source catalog  \citep[3FGL, ][]{2015ApJS..218...23A}. 
It contains sources that are significantly detected above 100 MeV and spans the first 4 years of the \Fermi mission. 
%A complete description of the 3FGL can be found in \citep{2015ApJS..218...23A}. .
We used the 3FGL catalog to search for PBH candidates and to constrain the local PBH evaporation rate.

To find PBH candidates in the 3FGL catalog, we first excluded from further consideration point sources associated with known astrophysical sources (such as blazars). Of 3033 sources in the 3FGL catalog, 1010 are unassociated sources. We also excluded sources that are within $10^\circ$ of the Galactic plane, which removed a further 468 sources. Our analysis was restricted to high-latitude sources because (a) detectable PBHs are expected be distributed isotropically given the detectability distances estimated in Section \ref{sec:sens} while astrophysical sources are concentrated along the Galactic plane, and (b) association of extragalactic sources such as blazars is easier at high latitude 
\citep[see, e.g.][]{2015ApJ...810...14A}.

To test the remaining unassociated 3FGL sources as PBH candidates, we fit their spectra with the time-integrated gamma-ray spectra emitted by a PBH.
For this analysis we used the fluxes of the candidate sources as reported in the 3FGL, which are provided in 5 energy bands (0.1 -- 0.3, 0.3 -- 1, 1 -- 3, 3 -- 10, and 10 -- 100 GeV). The time-integrated PBH spectrum depends on two parameters: initial mass (or temperature) and the distance to the Earth (equivalent to an overall normalization).
We varied these two parameters to obtain the best fit to the PS spectrum in the five energy bands. The quality of the spectral fit to the reported spectrum was determined by calculating the value of the $\chi ^2$ over the five energy bands:

\noindent
\begin{equation}
\chi^2 = \sum_{i = 1}^{5} \frac{(\Phi_i-\Psi_i )^2}{\sigma_i^2}, 
\end{equation}
where $\Phi_i$ is the PBH spectrum integrated over the width of bin $i$ and $\Psi_i$ is the flux in bin $i$ from the 3FGL. Here $\sigma_i$ represents the uncertainty on the flux in bin $i$.
In the 3FGL, flux uncertainty is represented by a 68\% confidence interval; the value of the uncertainty can then be written as the difference between the best-fit flux and either the upper or lower bound.
We set $\sigma_i$ to be the larger of the two in order to be conservative.
We require the value of the best-fit $\chi^2$ to be below the critical value of 11.3, which corresponds to 99\% exclusion for 5 degrees of freedom and 2 parameters. In other words, sources with a $\chi ^2$ value greater than 11.3 have only a 1\% chance of being spectrally consistent with a PBH. After the spectral consistency was computed, 318 sources out of the 542 unassociated candidates remained as PBH candidates.

The candidate sources next underwent a check for proper motion. 
We use the following algorithm to determine the magnitude and significance of proper motion:
\begin{enumerate}
\item
All source-class photons above 1 GeV within $5^\circ$ of the source's reported 3FGL location were collected.
The time range (August 2008 to July 2012) and data reconstruction (\texttt{P7REP\char`_SOURCE\char`_V15}) were consistent with that of the data used to construct the 3FGL.
Since the angular resolution of the \Fermi LAT decreases quickly below 1 GeV, including photons below 1 GeV did not have a significant impact on the final results.
\item 
Data covering a longer time range (August 2008 to July 2017) and a more recent event reconstruction (\texttt{P8R2\char`_SOURCE\char`_V6}) were held in reserve for validation, and was used to test the PBH hypothesis for any sources that passed the proper motion cut.

\item
The expected number of photons $N$ from the source of interest was calculated by multiplying the flux in each energy bin by 
the \Fermi-LAT exposure at the bin's midpoint energy, and summing over the three relevant bins (1 -- 3 GeV, 3 -- 10 GeV, 10 -- 100 GeV). 
\item 
\label{step:prop_motion_likelihood}
In order to estimate the velocity of a PS we compare the maxima of the likelihood function $\mathcal{L}(\vec{x_i},t_i,\vec{x_0},\vec{v_0})$ in two cases: 
fixed $\vec{v_0}=0$ and free $\vec{v_0}$. 
We approximate the point spread function of \Fermi LAT by a Gaussian for simplicity.
The likelihood function is given by multiplying over $N$ photons around the initial position of the source:

\noindent
\begin{equation}
\label{eq:vlike}
\mathcal{L} = \prod_{i=1}^{N}w_i\times \textup{exp} \big\{\frac{{-(\vec{x_i}-\vec{x_0}-\vec{v_0}t_i)^2}}{{\sigma_i^2}}\big\}
\end{equation}
where $\vec{x_i}$ is the coordinates of the photon, $\vec{v_0}$ is the proper motion of the source, $t_i$ is the photon arrival time, $\vec{x_0}$ is the source location at the beginning of the observation time, and $\sigma_i$ is the 68\% angular containment radius for a photon at energy $E_i$, which is $\sim 0.7^\circ$ at 1 GeV \citep{2013arXiv1304.5456B}.
Here $w_i$ is a weight assigned to each photon, the calculation of which is defined in Section \ref{sec:likelihood}. 

In practice, we use the natural logarithm of the likelihood:
\begin{equation}
\label{eq:vloglike}
\log \mathcal{L} = \sum_{i=1}^{N}\log w_i-\frac{{(\vec{x_i}-\vec{x_0}-\vec{v_0}t_i)^2}}{{\sigma_i^2}}
\end{equation}

The main difficulty is separating the $N$ photons attributed to the source from background photons.
Our algorithm chooses a 4-dimensional grid of points around an initial value of $\vec{x_0}$ and $\vec{v_0} = 0$, and for each grid point finds the $N$ photons
inside the $5^\circ$ ROI that have the highest contribution to $\log \mathcal{L}$, i.e. the photons which most likely belong to the source given a particular position and velocity.
Therefore the weights $w_i$ do not appear as a prefactor in Eqs \ref{eq:vlike} and \ref{eq:vloglike} because the $N$ best-fit photons change given different assumptions of $\vec{x_0}$ and $\vec{v_0}$.
The best-fit $\vec{x_0}$ and $\vec{v_0}$ are found by maximizing $\Delta \log \mathcal{L} = \log \mathcal{L} - \log \mathcal{L}(\vec{v_0}=0)$ on the grid.
With the additional degrees of freedom from allowing $\vec{v_0}$ to float, the value of $\Delta \log \mathcal{L}$ is always nonnegative. 

We find in MC simulations (described in Section \ref{sec:MClimit}) that this algorithm tends to underestimate the input velocity by $\approx 25\%$;
the best-fit velocity should therefore be considered a lower bound on the true velocity and sufficient for our purpose of separation of moving and stationary sources.
The underestimation occurs because source photons that are far away from the average source position have lower weights and so are less often included in the likelihood calculation.
In their stead are background photons, whose distribution in time is random, and therefore cause the algorithm to favor a slower overall velocity.
%The $N$ photons used in the calculation will generally change over the course of the maximization procedure, because it is not possible to discriminate between individual source and nearby background photons. However, the algorithm converges if the source is sufficiently brighter than the background. 
\item 
The significance of $\Delta \log \mathcal{L}$ for each source was found by assigning random times $t_i$ drawn from a flat distribution to each photon but fixing the positions $\vec{x_i}$ of all the photons, and reoptimizing $\Delta \log \mathcal{L}$. 
This process was repeated 50 times for each source, and the original value of $\Delta \log \mathcal{L}$ was compared with the distribution of $\Delta \log \mathcal{L}$ for the data sets scrambled in time to find a local significance $\sigma$:

\noindent
\begin{equation}
\sigma = \frac{\Delta \log \mathcal{L}_0 - \overline{\Delta \log \mathcal{L}_s}}{\textup{std}(\Delta \log \mathcal{L}_s)},
\end{equation}
where $\Delta \log \mathcal{L}_0$ is the original value of the improvement in likelihood, $\overline{\Delta \log \mathcal{L}_s}$ is the mean of the scrambled likelihood improvements, and $\textup{std}(\Delta \log \mathcal{L}_s)$ is the standard deviation of the scrambled likelihood improvements.
\item
A cut on the local significance for each source was made at $3.6 \sigma$ which corresponds to a global significance of $2 \sigma$ for 318 sources. 
\end{enumerate}
A single source (3FGL J2310.1$-$0557, see Section \ref{sec:j2310} for a discussion of this source) exceeded this cut on local significance, and the standard deviation of the local significances of the entire set of candidates was 1.03, which is consistent with statistical fluctuations. After examining the data held in reserve for J2310.1$-$0557 (described in Section \ref{sec:j2310}), we concluded that no likely PBH candidates exist in the 3FGL catalog.

%\begin{figure}[htbp]
%\begin{center}
%\epsfig{figure = motion_distribution.pdf, scale=\onepic}
%\noindent
%\caption{\small 
%\label{fig:motion_dist}
%Measurement of proper motion for unassociated 3FGL sources with fluxes consistent with PBH evaporation.
%Red solid line shows the normal distribution.
%The vertical dashed line corresponds to a global significance of $2.0\sigma$. No sources fall to the right of this line.
%}
%\end{center}
%\end{figure}

\subsection{Calculating Photon Weights}
\label{sec:likelihood}

The photon weights $w_i$ in equations \ref{eq:vlike} and \ref{eq:vloglike} are defined as the probability that a given photon originated from the candidate PS, and are calculated by performing a standard likelihood optimization with the \Fermi Science Tool {\tt gtlike}\footnote{Science Tools version v10r0p5, available at \url{http://fermi.gsfc.nasa.gov/ssc/data/analysis/software}}.
The model used includes all 3FGL PS within 5$^\circ$ of the candidate source, as well as the standard Pass 7 models for Galactic and isotropic diffuse emission.
The candidate source is modeled as an extended source with a radial Gaussian profile with $\sigma=0.25^\circ$ instead of a PS, in order to account for the possibility of proper motion.
The data were binned into three logarithmically spaced energy bands between 1 GeV and 100 GeV and in $0.1^\circ \times 0.1^\circ$ spatial pixels.
After the model was optimized by {\tt gtlike}, weights were assigned to each photon (described its coordinates $x$, $y$, and energy $E$) by calculating the fraction of the flux belonging to the candidate source in each pixel:

\begin{equation}
w_{x,y,E} = \frac{\Phi'(x,y,E)}{\sum_i \Phi_i(x,y,E)}
\end{equation}
where $\Phi'(x,y,E)$ is the predicted flux from the candidate source in the pixel and $\Phi_i(x,y,E)$ are the fluxes from all the sources in the model.
In addition, the 3FGL PS were masked by assigning a weight of zero to all the photons which fell in a pixel more than 1$^\circ$ from the candidate source position where the summed contribution of the non-candidate PS fluxes exceeded 10\% of the total flux in that pixel. This meant that all the photons in the calculation had a high probability of originating either from the candidate source or the diffuse background.

The weighting has little impact on the reconstruction of proper motion because individual photon weights do not change as the likelihood maximization from step \ref{step:prop_motion_likelihood} optimizes $\vec{x_0}$ and $\vec{v_0}$. However, weighting the photons in this way prevents the algorithm from interpreting photons from nearby sources as originating from the candidate source. Without weighting, we found that flaring nearby sources could mimic a moving source and therefore lead to false positive results.

\subsection{J2310.1$-$0557}
\label{sec:j2310}
The source J2310.1$-$0557 passed the proper motion cut with a significance of 4.2$\sigma$, and was therefore investigated further.
Approximately 9 years (August 2008 to July 2017) of Pass 8 (\texttt{P8\char`_SOURCE\char`_V6}) data above 1 GeV in an ROI of 5$^\circ$ around the source location were collected.
The increased statistics and improved angular resolution of the Pass 8 data set clearly indicated that J2310.1$-$0557 lies approximately 1$^\circ$ away from a separate, highly variable source of gamma rays which is not in the 3FGL catalog.
This source flared brightly (approximately 150 photons) on 2011 March 7, near the end of the 3FGL time period but was quiet for the remainder of the period.
We found that the position of the source was consistent with the Sun, which flared brightly on the same date \citep{2011ATel.3214....1A}.
gamma-ray emission from the Sun and Moon not included in our models of the ROI.
The effect of the solar flare near a candidate PBH was to mimic a moving source, which explains why the proper motion algorithm returned a positive result. 
Because the sources in the Monte Carlo simulation described in Section \ref{sec:MClimit} are placed at random points on the sky, we expect that similar false positives will occur in the simulations.
Therefore, we report the upper limit on PBH evaporation rate as if one source passed our criteria, even though J2310.1$-$0557 is not a good PBH candidate. Incidentally after the publication of the 3FGL source list, J2310.1$-$0557 was found to be a millisecond pulsar\footnote{See https://confluence.slac.stanford.edu/display/SCIGRPS/LAT+Pulsations+from+PSR+J2310-0555}.


\section{\Fermi-LAT limits on PBH\MakeLowercase{s}}
\label{sec:MClimit}

We used Monte Carlo (MC) simulations to derive the efficiency for detecting PBHs, and used the efficiency to place upper limits on the local PBH evaporation rate.
We generated a sample of PBHs within 0.08 pc of the Earth with uniform spatial density and random velocities with an average speed of
$250\rm\,km\, s^{-1}$, which is close to an upper bound on orbital velocity of the Sun around the Galactic center
\citep{1999MNRAS.310..645W, 2008ApJ...684.1143X, 2009PASJ...61..227S, 2010ApJ...720L.108G, 2010MNRAS.402..934M, 2011MNRAS.414.2446M},
and 3-dimensional velocity dispersion equal to the local velocity dispersion of dark matter, $270\rm\,km\, s^{-1}$ \citep{2010JCAP...02..030K}.
At the end of this section, we also derive the limits for different assumptions about the PBH distribution, such as the relative velocity and 
velocity dispersion, to estimate the corresponding systematic uncertainty.

We assume the PBH population has a constant rate of PBH evaporations,
$\dot{\rho}_{\rm PBH} = const.$
We also assume a uniform PBH density distribution in the vicinity of the Earth. 
A constant rate of evaporation implies that the derivative of the PBH density is related to the PBH temperature as:

\noindent
\begin{equation}
\label{eq:Tdistr}
\frac{d \rho_{\rm PBH}}{d T} \propto T^{-4}.
\end{equation}

%This relation can be derived as follows: Let $\rho_{\rm PBH}(T)$ be the density of PBHs with temperature larger than $T$. If $\tau(T)$ is the lifetime of a PBH with temperature $T$, then all PBHs with $T' > T$ will evaporate during time $\tau$. Consequently,

%\noindent
%\begin{equation}
%\label{eq:rho0}
%\rho_{\rm PBH}(T) = \int_0^{\tau(T)} \dot{\rho}_{\rm PBH} dt = \dot{\rho}_{\rm PBH}\; \tau(T).
%\end{equation}
%Now we take into account that $\tau \propto T^{-3}$ and differentiate with respect to $T$ to get
%Equation (\ref{eq:Tdistr}).

The following steps were performed in the derivation of the PBH evaporation rate limit:
\vspace{-2mm}
\begin{enumerate}
\item
\label{item:make_pbhs}
A sample of PBHs $(T_i, \vec{x}_i, \vec{v}_i)$ was simulated with temperatures $T_i>5\: {\rm GeV}$ and $T_i<60\: {\rm GeV}$ distributed according to Equation (\ref{eq:Tdistr}), and distances $R_i$ within $R < 0.08$ pc around the Earth. 
The velocities $v_i$ of the sample PBHs were distributed with mean equal to the orbital velocity of the Sun, 
$v_{\rm rot} = 250\; {\rm km\: s^{-1}}$,
and dispersion $v_{\rm disp} = 270\; {\rm km\: s^{-1}}$.
\item
For each PBH, we simulated the detection of the photons emitted over the 4 year 3FGL time period, consistent with the PBH evolution. The energies were distributed according to the instantaneous PBH spectrum of the appropriate temperature, and the positions of the photons were smeared according to the 
\Fermi-LAT point-spread function (modeled as a Gaussian distribution).
The emission spectra of PBHs $\Phi(E,t)$ are discussed in Appendix \ref{app:PBH_spectrum}. 
We used a time step of $\Delta t = 1$ day in modeling the evolution of the PBH position and temperature, 
and the number of photons detected by the \Fermi LAT each day was given by a Poisson random value with a mean of 

\noindent
\begin{equation}
\overline{N(t)} = \frac{ \Delta t}{4\pi R^2}\int_{\textup{E=100 MeV}}^{\textup{E=500 GeV}} {\Phi(E,t)} A(E) dE,
\end{equation}
where $A$ is defined as the average \Fermi-LAT exposure per unit time at the position of the simulated PBH, and $R$ is the distance from the Earth.
The energy of each photon was found by random sampling of $\Phi(E,t)\times A(E)$. 
\Fermi LAT has relatively uniform exposure on time periods longer than 1 day.
\item
\label{item:detection}
The list of simulated PBH photons was concatenated to the real photons present within $5^\circ$ of the final location of the PBH, with the same data selection as the 3FGL. A likelihood fit using the \Fermi Science Tool {\tt gtlike} was performed in a $7^{\circ}\times7^{\circ}$ ROI centered at the same location, using a model of the sky which included all 3FGL sources within $5^\circ$ of the ROI center, as well as models of the isotropic diffuse and Galactic diffuse emission\footnote{The models used were the standard Pass 7 (for consistency with the 3FGL) diffuse emission models available from \url{https://fermi.gsfc.nasa.gov/ssc/data/access/lat/BackgroundModels.html}}. The PBH was modeled as a source with a LogParabola spectrum, with fitting parameters restricted to the ranges $1.2<\alpha<3.0$ and $0.0<\beta<1.0$. Once the likelihood maximization was complete, the PBH was considered detected if its TS value was greater than 25, which is consistent with the 3FGL cutoff.
\item
\label{item:spectrum}
If the PBH source was detected, the results from the likelihood fit were used to find the source flux in the five energy bins reported in the 3FGL catalog. The spectral consistency with a PBH spectrum was then calculated in the same way as described in Section \ref{sec:search}.
\item 
\label{item:motion}
If the source was found to be spectrally consistent with a PBH, the significance of any proper motion was evaluated by the algorithm described in Section \ref{sec:search}. The combined efficiency of steps \ref{item:detection}$-$\ref{item:motion} is displayed in Figure \ref{fig:det_map}. We smoothed the results by convolving the detectability map with a 3$\times$3 matrix of ones, which had a minor ($\approx$ 8\%) impact on the resulting limit. The impact of fluctuations was quantified by observing the change in the limit as the number of simulations increased; we found that an increase of the number of simulations by 100\% had less than a 20\% change in the resulting limit.
\item 

\label{step:eff}
To derive an upper limit on the number of PBH evaporations in our search region, we begin with the number of expected detections:

\noindent
\begin{equation}
\label{eq:N_det}
N = \rho \epsilon V,
\end{equation}
where $\rho$ is the true density of PBHs and $V$ is the volume searched.
$\epsilon$ is the average PBH detection efficiency in time $t = 4$ yr and within the search volume $V$ (a sphere with radius 0.08 pc, with the wedge corresponding to $|b|<10^\circ$ removed); 
it is calculated by taking the mean over the pixels in Figure \ref{fig:det_map} with the weight $R^2 T^{-4}$:


\noindent
\begin{equation}
\epsilon = \frac{\iint \epsilon (R, T) \frac{R^2}{T^4} \,dR\,dT }{\iint \frac{R^2}{T^4}\,dR\,dT },
\end{equation}
where the integrals run over the space of parameters described in step \ref{item:make_pbhs}. 
Equation \ref{eq:N_det} can be inverted to find the PBH density $\rho$ as a function of the number of detections $N$, or the upper limit on $\rho$ given an upper limit on $N$. Given that one PBH candidate passed the selection criteria described in Section \ref{sec:search}, we set an upper limit $N < 6.64$, 
which is the 99\% confidence upper limit on the mean of a Poisson distribution with 1 observed event.

\item
\label{item:limit}
We convert the upper limit on $\rho$ to an upper limit on $\dot{\rho}$ by finding the fraction $f$ of PBHs that would have evaporated during the search time 
$t$. 
Given a time of observation of 4 years, we find that all PBHs with initial temperature above 16.4 GeV would evaporate. Therefore,

\noindent
\begin{equation}
f = \frac{\int_{\textup{16.4 GeV}}^{\textup{60 GeV}} T^{-4} \,dT}{\int_{\textup{5 GeV}}^{\textup{60 GeV}} T^{-4} \,dT}.
\end{equation}
We calculate the 99\% upper limit on $\dot{\rho}_{\rm PBH}$ to be:

\noindent
\begin{equation}
\dot{\rho}_{\rm PBH} < f\frac{6.64}{\epsilon V t} = 7.2 \times 10^3 \textup{ pc}^{-3} \textup{ year}^{-1}.
\end{equation}

\item
We estimated the systematic uncertainties arising from the uncertaintes in the PBH spectrum by varying the overall normalization of the PBH spectrum (see Appendix \ref{app:PBH_spectrum}) and varying the velocity distributions of the Milky Way disk and DM halo. We consider two scenarios (``aggressive" and ``conservative") which give the best and worst sensitivity, respectively. Steps \ref{item:make_pbhs}$-$\ref{item:limit} are then repeated to find the resulting limit. The parameters of the aggressive and conservative models, as well as the resulting limits, are listed in Table \ref{table:systematics}.
\begin{table}[h]
\begin{center}
\begin{tabular}{|c c c c c|}
\hline
Model & Spectrum Normalization & Orbital Velocity (${\rm km\: s^{-1}}$) & DM Halo Velocity (${\rm km\: s^{-1}}$) & Limit\\
\hline
Aggressive & $\frac{0.45}{0.35}$ & 100 & 150 & $4.8\times 10^3  \textup{ pc}^{-3}  \textup{yr}^{-1}$ \\
Conservative & $\frac{0.25}{0.35}$ & 300 & 350 & $15.3\times10^3 \textup{ pc}^{-3} \textup{ yr}^{-1}$ \\
\hline
\end{tabular}
\caption{Parameters used in estimation of systematic uncertainty. 
To be more conservative in the estimates of the systematic uncertainties,
we have tested the ranges of orbital velocities and the DM dispersion velocities which are larger than most of the values reported
in the literature
\citep{1999MNRAS.310..645W, 2008ApJ...684.1143X, 2009PASJ...61..227S, 2010ApJ...720L.108G, 2010JCAP...02..030K,
2010MNRAS.402..934M, 2011MNRAS.414.2446M}.}
%e.g., $\sim 200 - 280\: {\rm km / s}$ in \cite{2010MNRAS.402..934M}.}
%The range of uncertainty in the orbital velocity is somewhat more conservative than the range given in \citep{2010MNRAS.402..934M}}
\label{table:systematics}
\end{center}
\end{table}


The limit including the systematic uncertainties is

\noindent
\begin{equation}
\label{eq:ev_rate}
%\dot{\rho}_{\rm PBH} < (19.2\pm 1.5 \textup{[stat]} ^{+14.8}_{-9.6} \textup{[syst]}) \times 10^3 \textup{ pc}^{-3} \textup{ yr}^{-1}.
\dot{\rho}_{\rm PBH} < (7.2^{+8.1}_{-2.4} ) \times 10^3 \textup{ pc}^{-3} \textup{ yr}^{-1}.
\end{equation}
\end{enumerate}

%For comparison, if we assume no candidate source had passed our selection criteria, the limit would be $\dot{\rho}_{\rm PBH} < (5.0^{+5.6}_{-1.6} ) \times 10^3 \textup{ pc}^{-3} \textup{ yr}^{-1}$.
\begin{figure}[htbp]
\begin{center}
\epsfig{figure = figures/detectability_map.pdf, scale=\onepic}
\noindent
\end{center}
\caption{\small 
\label{fig:det_map}
Fraction of simulated PBHs which are detected as a point source with a spectrum compatible with a PBH evaporation spectrum and with significant proper motion. The detectability peaks for PBHs with initial temperatures above 16.4 GeV because the lifetime of a 16.4 GeV PBH is 4 years, which is the same as the observation period of the 3FGL. Few PBHs are detected past a distance of 0.05 pc or below 10 GeV.}

\end{figure}






